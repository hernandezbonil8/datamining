\documentclass[a4paper,12pt]{article}
\usepackage[utf8]{inputenc}
\usepackage[T1]{fontenc}
\usepackage[english]{babel}
\usepackage{graphicx}
\usepackage{amsmath, amssymb}
\usepackage{booktabs}
\usepackage{hyperref}
\usepackage{csquotes}

\usepackage[margin=1in]{geometry}

\title{Predicting Game Outcomes Using Machine Learning in Sports Analytics}
\author{Emerson Hernandez \and Toufeeq Sharieff \and Mani Vutukuri \and Michael Nguyen}
\date{April 6, 2025\\Professor: Professor Fang Yi Yu}

\begin{document}
\maketitle
\tableofcontents
\newpage

\section*{Abstract}
This project aims to predict the outcome of sports matches (win, loss, or draw) by analyzing historical football team data from the 2021--2022 season. We build a machine learning pipeline that leverages key performance metrics to generate reliable predictions. In addition, we introduce a head-to-head prediction feature, allowing for direct comparison between two teams. This work provides actionable insights for coaches and enhances fan engagement.

\section{Introduction}
\subsection{Problem Statement}
The goal of this project is to develop a predictive model that forecasts team performance using historical season-level data. We use features such as goals for (GF), goals against (GA), points per game (Pts/G), and matches played (MP) to determine whether a team is "high performing" (above median wins) or not. We further extend our analysis to predict outcomes for head-to-head comparisons.

\subsection{Motivation}
\begin{itemize}
    \item \textbf{Strategic Insights:} Coaches and managers can adjust game plans based on performance predictions.
    \item \textbf{Fan Engagement:} Accurate forecasts offer added excitement and support interactive platforms such as fantasy leagues.
    \item \textbf{Data-Driven Decisions:} Machine learning reduces guesswork by providing objective performance insights.
\end{itemize}

\subsection{Proposed Methods}
Our methodology consists of the following steps:
\begin{enumerate}
    \item Data Collection and Preprocessing: Clean and explore the Kaggle dataset.
    \item Feature Engineering: Create useful features such as goal difference and incorporate key season statistics.
    \item Model Development: Train baseline models (Logistic Regression, Decision Trees, K-Nearest Neighbors) to classify teams as high or low performing.
    \item Model Evaluation: Compare models using accuracy and other metrics.
    \item Head-to-Head Prediction: Extend the model to predict which team has a higher probability of being high performing.
\end{enumerate}

\section{Related Work}
Research in sports analytics has demonstrated the potential of machine learning for predicting game outcomes:
\begin{itemize}
    \item \textbf{NBA Outcome Prediction:} Prior studies have used decision trees and logistic regression to predict outcomes using features like field goal percentages and rebounds.
    \item \textbf{Soccer Match Prediction:} Research focused on soccer has shown that historical match data, including metrics like goals and possession, can yield significant predictive accuracy.
\end{itemize}

\section{Data and Methodology}
\subsection{Dataset Description}
The dataset, obtained from Kaggle, includes season-level statistics for European football teams from the 2021--2022 season. Key features include:
\begin{itemize}
    \item Goals For (GF)
    \item Goals Against (GA)
    \item Points per Game (Pts/G)
    \item Matches Played (MP)
\end{itemize}
The target variable is defined as whether a team's wins exceed the median of the league.

\subsection{Methodology}
\begin{enumerate}
    \item \textbf{Exploratory Data Analysis (EDA):} Generate summary statistics and visualizations (e.g., correlation matrix) to understand the data.
    \item \textbf{Preprocessing:} Handle missing values and normalize features.
    \item \textbf{Model Training:} Implement Logistic Regression, Decision Trees, and K-Nearest Neighbors, and evaluate their performance on the validation and test sets.
    \item \textbf{Head-to-Head Prediction:} Utilize the best performing model to compare two teams directly.
\end{enumerate}

\section{Results and Discussion}
\subsection{Model Performance}
The best model, determined by validation accuracy, was used for final evaluation on the test set. Detailed performance metrics including accuracy, precision, recall, and confusion matrices are presented in the supplementary materials.

\subsection{Head-to-Head Analysis}
By comparing the predicted probabilities of two teams being high performing, our model can indicate which team is likely to have a competitive edge.

\section{Conclusion and Future Work}
This project demonstrates the viability of using a machine learning pipeline to predict team outcomes and facilitate head-to-head comparisons in sports analytics. Future work might include incorporating additional match-level data, exploring ensemble methods, and building a more sophisticated user interface.

\bibliographystyle{plain}
\bibliography{references}

\end{document}
